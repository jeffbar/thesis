\chapter{Conclusiones}


\lettrine[lines=1]{B}asados en los resultados en las secciones anteriores, podemos describir la historia de formación estelar de UGC11680 como sigue: los primeros resultados, el mapa $SFH$ de la Figura \ref{sfh_ugc11680} nos muestran un apagado de formación estelar en las zonas centrales  de esta galaxia ya que el gradiente temporal de su densidad de masa superficial tiene un cambio abrupto a edades tempranas, en forma de corte, y conforme se avanza a edades cosmológicas  actuales este corte corre diagonalmente, lo que implica un apagado en formación a radios más grandes y por lo tanto a zonas más externas.

\bigskip

\noindent El perfil de densidad de masa radial a diferentes tiempos de la Figura \ref{perfil_radial} nos indica que en efecto, el apagado es interno (cambio en gradientes), comparado con promedios de diferentes tipos de galaxias. El ensamblaje de masa además se comporta en UGC11680 con lo encontrado en trabajos anteriores \citep{perez2013} y como se muestra en la Figura \ref{ensamblaje_log}. Por lo tanto, la galaxia acumuló masa de dentro hacia afuera y el apagado comienza internamente, lo que también se muestra en su tasa de formación estelar temporal.

\bigskip

\noindent En todos los casos, el registro fósil señala crecimiento dentro-fuera y apagado de la misma forma, además de gradientes radiales que señalan que las partes internas tienen procesos que detienen este proceso de acumulación de masa estelar. Los promedios en cambio, muestran ensamblajes encontrados en otros trabajos \cite{perez2013} ({\color{red} Ibarra-Medel, et.al. (Submitted) }).

\bigskip

\noindent Los promedios de la Figura \ref{CMD} muestran también que la masa es una variable fundamental en el proceso de acumulación de masa temporal ya que los mapas  $SFH$ promediados por color-masa muestran una tendencia de formación estelar: entre más masivas , más rápido ensamblan estrellas  lo que no se muestra en las galaxias menos masivas y azules: estas siguen formando estrellas, según su $SFH$ promediado.

\bigskip


\noindent Entonces ¿De que forma encaja la historia de ensamblaje de masa de UGC11680 con respecto a las demás galaxias de la muestra? Todo parece indicar con las distribuciones de la Figura \ref{chi_todos_mean} y los resultados de la tabla \ref{tab_LN2} que UGC11680 encaja con las galaxias  rojas y de masa intermedia (las galaxias con masa  $\sim 10^{10} M_{\odot}$, con colores entre $3<g-r<4$ ), con el grupo de AGNs, las galaxias de masa intermedia en el valle verde  ($\sim 10^{10} M_{\odot}$, con colores entre $1<g-r<2$ y $2<g-r<3$ ) y  las galaxias masivas en el valle verde ($\sim 10^{11} M_{\odot}$ colores entre $2<g-r<3$) . 


\bigskip

\noindent Estos resultados son interesantes ya que entonces tenemos evidencia espacialmente resuelta de que UGC11680, siendo una galaxia espiral roja, ensambló su masa de dentro hacia afuera, con un apagado abrupto de la misma forma y que su historia de formación estelar es parecida o coincide mejor con galaxias de su masa y color, pero tambien con galaxias con AGN asi como las que se encuentran en el valle verde. 
\bigskip


\noindent También es interesante notar que el ajuste de la sección de comparación por medio de la distribución $\chi^{2}_{\nu}$ la masa parece ser un factor importante para que UGC11680 ensamblara y apagara su formación estelar rapidamente, ya que comienza con un ajuste con las galaxias de masa intermedia, seguidas de las galaxias más masivas, lo que nos indica que la masa es determinante en la formación y posterior ensamblaje de UGC11680, algo encontrado en trabajos anteriores para galaxias de la muestra de \textbf{CALIFA} \citep{perez2013}. 

\bigskip

\noindent Esto claro, no implica que sólo el AGN haya contribuido para el apagado de la formación estelar de UGC11680, sino que pudo iniciarlo para que procesos seculares como alguna interacción no violenta ó algún otro proceso físico haya continuado con el proceso, debido a que la galaxia siguió formando estrellas, pero a una tasa baja, lo que la colocó debajo de las galaxias SF pero sobre las galaxias retiradas  de la Figura \ref{sfr_califa}. 

\bigskip

\noindent Finalmente, por el análisis en la sección de poblaciones estelares, podemos excluir el polvo como causante de su color UGC11680:  entonces, debido al apagado rápido en formación estelar desde dentro hacia afuera, la galaxia pasó por el valle verde y  se situó en su posición dentro de la secuencia roja sólo por formación estelar promedio que mide el diagrama color-magnitud. Esto nos indica que UGC11680 es una galaxia peculiar con un ensamblaje de masa de dentro hacia afuera y que su color en el óptico resulta de una combinación de procesos internos de apagado abrupto en las zonas centrales, aunque sus poblaciones espacialmente resueltas indican que este apagado no fue total, sino de alguna forma regulado desde dentro hacia afuera, ya que la galaxia continua formando estrellas aunque a  una tasa baja a diferentes épocas cosmológicas en su historia de formación estelar. 




