
% Thesis Abstract -----------------------------------------------------


%\begin{abstractslong}    %uncommenting this line, gives a different abstract heading
\begin{abstracts}        %this creates the heading for the abstract page

En el catálogo del muestreo \textbf{CALIFA} (Calar Alto Legacy Integral Field Area  (\citet{sanchez2012}) hay una galaxia espiral roja peculiar llamada UGC11680. Esta galaxia esta a un redshift de $z \sim 0.026198$ (esto es a una distancia de $\sim$ 113 Mpc)  clasificada como una SBb Seyfert 2 (\citet{blazquez2007}) y esta situada en el diagrama Color-Magnitud en la zona conocida como ``secuencia roja'' normalmente habitada por grandes galaxias elípticas rojas, donde sabemos por dicho color, que ya no forman estrellas lo que la ubica dentro de las galaxias que sufrieron un apagado en su formacion estelar y que sin embargo conservan su morfología espiral. 

Los procesos físicos que dan lugar a este apagado, (conocido en la literatura como ``quenching''\footnote{La palabra \textsl{Quenching} no tiene una traducción que denote su significado en Inglés, por lo que se mantendrá el uso de la palabra, pero seremos cuidadosos en establecer claramente los intervalos temporales con los que delimitaremos su rango.}) siguen controversiales, por lo que se encuentran dentro de los temas sin concenso  en la astronomía extragaláctica actual. Por un lado, algunos autores señalan procesos externos que  ``desnudan'' a la galaxia de gas fresco para la formación de estrellas nuevas  (\citet{salim2007}, \citet{noeske2007},\citet{peng2010},\citet{dimatteo2005}) otros autores señalan factores internos que apagan la formación estelar \citet{martin2007}, \citet{nandra2007}, \citet{schawinski2007}).

Para poder entender que sucedió con  la historia de 
formación estelar (SFH) de   UGC11680, analizamos sus poblaciones estelares por medio de
datos de la conocida espectroscopía de campo integral. Para esto, utilizamos el pipeline \textbf{PIPE3D} y el código \textbf{FIT3D} (\url{http://www.astroscu.unam.mx/~sfsanchez/FIT3D/index.html}), diseñada para el análisis de cubos de datos resultado de \textbf{CALIFA}. De este novedoso análisis obtenemos imágenes y espectros  espacialmente resueltos y del resultado del proceso de datos, analizamos  
los mapas de historia de formacion estelar (\citet{cid2013_1}), y que arrojaron una clara evidencia del \textsl{quenching}
de la  galaxia de prueba en un tiempo dado. Igualmente, encontramos que el ensamblaje historico de masa tiene una formación "dentro-fuera" lo que coloca a la masa como una variable fundamental en el proceso de apagado.

Finalmente, la SFH de UGC11680  es comparada con los 574 historiogramas  de la muestra de \textbf{CALIFA}, separados y clasificados en sus valores color-masa usando la medida $\chi^2_{\nu}$ \footnote{Dado que hablamos de una medida reducida, es decir, una $\chi^2$ dividida por el número de grados de libertad $\nu$ definimos ``parecido '' a el número mas cercano a 1,y por lo tanto una $\chi^2_{\nu} =1 $ denotaría un parecido exacto} y encontrando que la historia de formacion estelar de UGC11680 es una combinacion en ``parecido'' 
con el grupo de AGNs  y de aquellas galaxias que se encuentran en el ``valle verde'' del diagrama color-masa, que se supone un espacio de transición 
entre la secuencia roja y la nube azul. Estos resultados, nos dan una idea de la importancia de la relación entre el AGN y su galaxia huesped
 asi como de la masa como variables importantes en el control de la tasa de formacion estelar en esta galaxia y probablemente en otras galaxias espirales rojas 
con caracteristicas similares.


\end{abstracts}
%\end{abstractlongs}


% ----------------------------------------------------------------------
